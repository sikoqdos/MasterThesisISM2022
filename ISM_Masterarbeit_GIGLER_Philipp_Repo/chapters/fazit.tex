\setlength{\parindent}{0em} 

%**********************************************************************************
% Kapitel Fazit und Zusammenfassung
%**********************************************************************************
\chapter{Fazit und Zusammenfassung}
\label{cha:fazit_und_zusammenfassung}
Die vorliegende Arbeit gibt einen Überblick über geltende IT-Sicherheitsanforderungen an eine Bank in Österreich. Im Zuge der Ausarbeitung wurden unterschiedliche regulatorische Anforderungen von Aufsichtsbehörden in einer Anforderungsmatrix dargestellt und gegliedert. In weiterer Folge wurden die relevanten Anforderungen an IT-Sicherheit untersucht und auf Basis ihrer technischen und organisatorischen Umsetzbarkeit aufgeteilt. Für die technisch umsetzbaren Anforderungen an IT-Sicherheit wurden Maßnahmen abgleitet und diese nach passenden Themengebieten gruppiert. Die einzelnen Maßnahmen jedes Themengebietes wurden beschrieben und deren Relevanz aufgezeigt. Im Zuge der Beschreibung der Maßnahmen wurden Best-Practice-Ansätze nach aktuellem Stand der Technik abgeleitet und beschrieben. Eine Qualitätsüberprüfung der abgeleiteten Best-Practice-Ansätze mittels der CDM rundet die Arbeit ab.
\bigbreak
Im Zuge der Ausarbeitung der relevanten Richtlinien hat sich gezeigt, dass 417 technische und organisatorische Anforderungen an IT-Sicherheit einer Bank in Österreich bestehen. Von diesen Anforderungen lassen sich 269 Anforderungen organisatorisch und 148 Anforderungen technisch umsetzen. Diese Aufteilung zeigt, dass das Thema IT-Sicherheit zu einem großen Teil organisatorische Maßnahmen erfordert. Die 148 technischen Anforderungen lassen sich mit Hilfe von 21 Maßnahmen umsetzen. Bei der Ableitung der Maßnahmen hat sich herausgestellt, dass sich ein Drittel der technischen Anforderungen mit der Behandlung der drei Themengebiete \glqq{}Identity-Management\grqq{}, \glqq{}Vulnerability-Management\grqq{} und \glqq{}Security Incident und Event Monitoring\grqq{} umsetzen lassen. Die technisch erforderlichen Maßnahmen sind in Tabelle \ref{table:maßnahmen_uebersicht_anzahl} aufgelistet. Im Zuge der weiteren Ausarbeitung konnten 10 Best-Practice-Ansätze definiert werden, mit deren sich die 21 erforderlichen Maßnahmen umsetzen lassen. Die Best-Practice-Ansätze und die darin enthaltenen Maßnahmen sind in Abbildung \ref{fig:bp-matrix} dargestellt. Abbildung \ref{fig:matrix_ueberpruefung} zeigt, dass eine erfolgreiche Umsetzung der Maßnahmen und damit einhergehend eine Erfüllung der technischen Anforderungen mittels technischer und organisatorischer Mittel gemessen werden kann. 
\bigbreak
Die Arbeit zeigt, dass geltende technische IT-Sicherheitsanforderungen an eine Bank in Österreich auf Basis von Best-Practice-Ansätzen erfüllt werden können. Es ist darauf zu achten, dass man sich bei der Implementierung von technischen IT-Sicherheitsmaßnahmen nicht nur auf die von Aufsichtsbehörden vorgeschrieben Anforderungen beschränken sollte. Aufsichtsbehörden geben nur ein Rahmenwerk vor, die Vollständigkeit der durch die Richtlinien abgeleiteten Maßnahmen darf nicht als gegeben angenommen werden. Schlussendlich geht es nicht darum die Anforderungen von Aufsichtsbehörden zu erfüllen, sondern die IT-Sicherheit des Unternehmens sicherstellen zu können und so gegen bösartige Angriffe gewappnet zu sein.