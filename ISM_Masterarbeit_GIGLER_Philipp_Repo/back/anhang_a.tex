\chapter{Peer Review}
\label{app:Peer Review}

\section{Peer 1}
\subsubsection{Fragestellung 1 - Aufsichtsbehörden}
\begin{itemize}
    \item Welchen Aufsichtsbehörden müssen Sie im Rahmen von regulatorischen Überprüfungen der Bank in Bezug auf IT-Sicherheit Rede und Antwort stehen?\\
    Antwort: Im Lead, wenn es um Prüfungen geht ist die FMA.\\
    \item Welchen Aufsichtsbehörden gegenüber sind sie im Falle eines Security-Incidents innerhalb der Bank meldepflichtig?\\
    Antwort: Security-Incidents gehen zentral an die FMA, es gibt jedoch auch EZB Meldepflichten, meines Wissens aber nur für Unternehmen, welche auch direkt von der ETB überwacht werden (Großbanken). \\
    \item Sind die in Tabelle \ref{table:uebersicht_evaluierte_literatur} aufgelisteten Aufsichtsbehörden Ihrer Meinung nach relevant für eine Bank in Österreich?\\
    Antwort: EIOPA gilt nur für Versicherungen, nicht für Banken und bildet das Pendant zu den EBA-Guidelines. Die EU ist in diesem Sinne keine Aufsichtsbehörde. Die Überwachung wird von beauftraten Behörden durchgeführt (EZB oder FMA), von der EU wird nur der Rahmen vereinbart bzw. vorgegeben. Die ENISA ist in diesem Sinne auch keine Behörde sondern eine Agentur, die im Auftrag der EU öffentliche Institutionen sowie Behörden unterstützt. \\
    \item Ist die Auflistung relevanter Aufsichtsbehörden für eine Bank in Österreich in Tabelle \ref{table:uebersicht_evaluierte_literatur} vollständig?\\
    Antwort: Ja, die Auflistung ist vollständig.\\
    \item Falls nicht, welche relevanten Aufsichtsbehörden fehlen?
\end{itemize}
\subsubsection{Fragestellung 2 - Richtlinien}
\begin{itemize}
    \item An welchen Richtlinien orientiert sich Ihr Unternehmen bei der Etablierung von IT-Sicherheitsanforderungen bzw. bei der Erstellung von Governance-Richtlinien?\\
    Antwort: Primär an Hand der Vorgaben der FMA\\
    \item Sind die in Tabelle \ref{table:uebersicht_evaluierte_literatur} aufgelisteten Richtlinien Ihrer Meinung nach relevant für eine Bank in Österreich?\\
    Antwort: Ja, die Richtlinien und Vorgaben sind relevant.\\
    \item Ist die Auflistung relevanter Richtlinien in Tabelle \ref{table:uebersicht_evaluierte_literatur} vollständig?\\
    Antwort: Ja, die Auflistung ist vollständig.\\
    \item Falls nicht, welche relevanten Richtlinien fehlen?
\end{itemize}
\bigbreak

\section{Peer 2}
\subsubsection{Fragestellung 1 - Aufsichtsbehörden}
\begin{itemize}
    \item Welchen Aufsichtsbehörden müssen Sie im Rahmen von regulatorischen Überprüfungen der Bank in Bezug auf IT-Sicherheit Rede und Antwort stehen?\\
    Antwort: FMA\\
    \item Welchen Aufsichtsbehörden gegenüber sind sie im Falle eines Security-Incidents innerhalb der Bank meldepflichtig?\\
    Antwort: FMA\\
    \item Sind die in Tabelle \ref{table:uebersicht_evaluierte_literatur} aufgelisteten Aufsichtsbehörden Ihrer Meinung nach relevant für eine Bank in Österreich?\\
    Antwort: Ja, ggf. ist die DSGVO zu ergänzen. EIOPA ist für Versicherungen relevant (Hier stellt sich die Frage, welche Produkte von der Bank angeboten werden).\\
    \item Ist die Auflistung relevanter Aufsichtsbehörden für eine Bank in Österreich in Tabelle \ref{table:uebersicht_evaluierte_literatur} vollständig?\\
    Antwort: Ja\\
    \item Falls nicht, welche relevanten Aufsichtsbehörden fehlen?
\end{itemize}
\subsubsection{Fragestellung 2 - Richtlinien}
\begin{itemize}
    \item An welchen Richtlinien orientiert sich Ihr Unternehmen bei der Etablierung von IT-Sicherheitsanforderungen bzw. bei der Erstellung von Governance-Richtlinien?\\
    Antwort: FMA\\
    \item Sind die in Tabelle \ref{table:uebersicht_evaluierte_literatur} aufgelisteten Richtlinien Ihrer Meinung nach relevant für eine Bank in Österreich?\\
    Antwort: Das BAIT-Rundschreiben ist für Österreich nicht relevant, alle anderen sehe ich als relevant an. \\
    \item Ist die Auflistung relevanter Richtlinien in Tabelle \ref{table:uebersicht_evaluierte_literatur} vollständig?\\
    Antwort: Ja\\
    \item Falls nicht, welche relevanten Richtlinien fehlen?
\end{itemize}
\bigbreak

\section{Peer 3}
\subsubsection{Fragestellung 1 - Aufsichtsbehörden}
\begin{itemize}
    \item Welchen Aufsichtsbehörden müssen Sie im Rahmen von regulatorischen Überprüfungen der Bank in Bezug auf IT-Sicherheit Rede und Antwort stehen?\\
    Antwort: Wir werden von der FMA auditiert und melden auch Security-Incidents an die FMA.\\
    \item Welchen Aufsichtsbehörden gegenüber sind sie im Falle eines Security-Incidents innerhalb der Bank meldepflichtig?\\
    Antwort: Security-Incidents werden in erster Instanz an die FMA gemeldet.\\
    \item Sind die in Tabelle \ref{table:uebersicht_evaluierte_literatur} aufgelisteten Aufsichtsbehörden Ihrer Meinung nach relevant für eine Bank in Österreich?\\
    Antwort: Für eine nationale Bank in Österreich ist die BaFin nicht relevant. Die Inhalte der BaFin werden häufig von der FMA übernommen. Die EIOPA ist für eine Bank nicht relevant. Die ENISA ist keine Aufsichtsbehörde aber natürlich relevant für Banken in Österreich.\\
    \item Ist die Auflistung relevanter Aufsichtsbehörden für eine Bank in Österreich in Tabelle \ref{table:uebersicht_evaluierte_literatur} vollständig?\\
    Antwort: Ja, ich habe nichts zu ergänzen.\\
    \item Falls nicht, welche relevanten Aufsichtsbehörden fehlen?
\end{itemize}
\subsubsection{Fragestellung 2 - Richtlinien}
\begin{itemize}
    \item An welchen Richtlinien orientiert sich Ihr Unternehmen bei der Etablierung von IT-Sicherheitsanforderungen bzw. bei der Erstellung von Governance-Richtlinien?\\
    Antwort: Wir orientieren uns hauptsächlich an den Vorgaben der FMA.\\
    \item Sind die in Tabelle \ref{table:uebersicht_evaluierte_literatur} aufgelisteten Richtlinien Ihrer Meinung nach relevant für eine Bank in Österreich?\\
    Antwort: BAIT-Rundschreiben ist nicht relevant, kann aber als Quelle herangezogen werden. Die Inhalte decken sich mit den Inhalten der FMA.\\
    \item Ist die Auflistung relevanter Richtlinien in Tabelle \ref{table:uebersicht_evaluierte_literatur} vollständig?\\
    Antwort: Aus meiner Sicht, ja.\\
    \item Falls nicht, welche relevanten Richtlinien fehlen?
\end{itemize}
\bigbreak
