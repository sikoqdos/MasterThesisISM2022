%Einrücken bei Absatz verhindern
\setlength{\parindent}{0em} 

%**********************************************************************************
% Kapitel Stand des Wissens / Stand der Technik
%**********************************************************************************
\chapter{Stand des Wissens}
\label{cha:stand_des_wissens_stand_der_technik}
Ziel dieses Kapitels ist es, einen Überblick über die Begriffe \glqq{}Governance\grqq{}, \glqq{}Compliance\grqq{} und \glqq{}IT-Sicherheit\grqq{} zu geben, Unterschiede aufzuzeigen und so zu einem besseren Verständnis der weiteren Arbeit beizutragen. Im Zuge dessen wird auf die Relevanz von IT-Sicherheit im Banken-Sektor eingegangen und die möglichen Auswirkungen von IT-Sicherheitsvorfällen im Bankenumfeld aufgezeigt. 
\bigbreak
Der Begriff IT-Sicherheit beschreibt unterschiedliche Technologien, Prozesse und Praktiken, deren Zweck der Schutz von Assets in einem Unternehmen vor unberechtigten Zugriff oder unrechtmäßiger Verwendung ist. Unternehmen haben verschiedene Motivationsgründe und Ausprägungen zum Schutz ihrer Assets und Unternehmensdaten. Unabhängig vom jeweiligen Asset lassen sich drei Schutzziele ableiten, die in der Literatur als „CIA Triade“ bezeichnet werden  \autocite{DuttaNitul2021CS}:
\begin{itemize}
    \item (C) Confidentiality\\
    Das erste Schutzziel der CIA-Triade bildet die „Vertraulichkeit“ von Assets und Daten. Das Ziel ist der Schutz vor unberechtigtem Zugriff auf das zu schützende Asset. Vertrauliche Daten sollen nur von autorisierten Personen oder Systemen eingesehen werden können. 
    \item (I) Integrity\\
    Das zweite Schutzziel bildet die „Integrität“ von Assets und Daten. Integrität verlangt, dass sowohl die Daten selbst als auch deren Funktionsweise zu jedem Zeitpunkt korrekt sind. Änderungen an Assets oder Daten müssen stets nachvollziehbar sein. 
    \item (A) Availability\\
    Das dritte Schutzziel bildet die „Verfügbarkeit“ von Assets und Daten. Darunter ist zu verstehen, dass Assets und Daten zu jedem Zeitpunkt verfügbar sein müssen. Es ist zu vermeiden, dass Daten verloren gehen oder der Zugriff auf Assets nicht gegeben ist. 
\end{itemize}
\bigbreak
Angriffe auf IT-Systeme zielen auf die Verletzung eines oder mehrerer Schutzziele ab. Je nach Ausprägung eines Unternehmens kann die Verletzung eines Schutzziels unterschiedliche Auswirkungen haben. Vor allem im Bankensektor, dem Dreh- und Angelpunkt internationaler Geldtransaktionen steht der Schutz von Assets und Daten an erster Stelle. 
\bigbreak
Dagegen steht die steigende Anzahl an Angriffen auf Bankinstitute. Im Jahr 2019 liesen sich fast 50\% aller untersuchten Phishing-Angriffe auf den Bankensektor zurückzuführen. Auch Ransomware-Attacken auf Banken sind im Jahr 2020 im Vergleich zum Jahr 2019 um 520\% gestiegen. \autocite{crosman_2020}
\bigbreak
Auf Basis der stetig steigenden Bedrohungslage und als  Reaktion auf verschiedene Unternehmenspleiten in den 1990er Jahren, wie beispielsweise der Bearing-Bank oder von Worldcom, sehen sich Unternehmen mittels regulatorischen Vorgaben zu einer transparenten Unternehmensführung verpflichtet \autocite{capital_2022} \autocite{capital_20222}. Die Einhaltung dieser Vorgaben wird seitens Gesetzgebern und relevanten Aufsichtsbehörden verstärkt in den Fokus genommen. 
\bigbreak
Die Erfüllung dieser regulatorischen Anforderungen wird mit dem Begriff \glqq{}Compliance\grqq{} bezeichnet. Compliance leitet sich aus dem lateinischen Wort \glqq{}complere\grqq{} ab. Übersetzt bedeutet es ausfüllen beziehungsweise ergänzen. Im wirtschaftlichen Bereich wird der Ausdruck im übertragenen Sinne für die Übereinstimmung mit etwas oder dem Einhalten von geltendem Recht verwendet. Das Ziel, compliant gegenüber unterschiedlichen Anforderungen zu sein, bezeichnet den Zustand der Anforderungskonformität gegenüber gesetzlichen oder aufsichtsrechtlichen Vorgaben. Das Erreichen des Ziels, compliant gegenüber Anforderungen zu sein, wird in Unternehmen mittels \glqq{}Governance\grqq{} angestrebt. Der Begriff Governance leitet sich aus dem lateinischen Wort \glqq{}gubernare\grqq{} ab. Übersetzt bedeutet es so viel wie leiten, lenken oder steuern.  \autocite{FalkMichael2012IidC} 
\bigbreak
\textcite{JohannsenWolfgang2011RfI:} beschreibt den Begriff Governance mit der \glqq{}verantwortlichen, transparenten und nachvollziehbaren Leitung und Überwachung von Organisationen und ihren Ausrichtungen an Regulierungen, Standards und ethischen Grundsätzen\grqq{}. Im wirtschaftlichen Bereich wird der Ausdruck als Synonym für die Leitung und Überwachung von Unternehmen verwendet. Der Begriff Governance lässt sich in verschiedene Teilbereiche gliedern. Bezogen auf diese Arbeit sind die beiden Teilbereiche Corporate-Governance und IT-Governance relevant. \textcite{hauschka} beschreiben Corporate-Governance mit dem Begriff \glqq{}Unternehmensverfassung\grqq{} und beziehen sich damit auf einen Ordungsrahmen für die Leitung und Überwachung in einem Unternehmen. 
\bigbreak
Mit Hilfe von Corporate-Governance können Managementsysteme geschaffen werden, die mittels internen Überwachungsmechanismen die Transparenz von Abläufen in Unternehmen erhöhen. Eine Möglichkeit für interne Überwachungsmechanismen sind \glqq{}Interne Kontrollsysteme\grqq{} (IKS), welche die Unternehmensleitung bei der Erkennung von Risiken unterstützten. Ein IKS entspricht einer Ansammlung von Maßnahmen, sogenannten Controls, die in betriebliche Prozesse integriert werden können. Die Controls umfassen Richtlinien und Verfahren in Abläufen des Unternehmens und werden dafür verwendet, unerwünschten Ergebnissen vorzubeugen, Risiken zu erkennen und diese zu adressieren. Der Begriff \glqq{}IT-Governance\grqq{} wird in der Praxis verwendet um Themen aus dem Bereich Governance im IT-Umfeld zu adressieren oder ein Regelwerk beziehungsweise Kontrollen für IT-Systeme zu  etablieren. \autocite{FalkMichael2012IidC}