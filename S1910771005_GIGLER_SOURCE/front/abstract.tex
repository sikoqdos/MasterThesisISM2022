\chapter{Abstract}
Due to the increasing threat of cyber attacks on IT systems, regulatory authorities are increasingly imposing requirements on companies' IT security. A cyber attack or a failure of IT infrastructure components can cause considerable financial and reputational damage to companies. In particular, the security of banks and other financial service providers is increasingly in the focus of supervisory authorities. On the one hand, financial institutions need to increase the technical security of their own IT environment in order to thwart attacks, and on the other hand, they need to meet the various requirements of supervisory authorities. This paper deals with the requirements of supervisory authorities for the technical IT security of banks in Austria. In the course of this work, relevant guidelines are collected and the completeness of the examined guidelines is ensured by means of peer reviews. The guidelines are then analyzed in terms of technical security requirements that need to be implemented. These security requirements are categorized and possible technical implementations are derived. The question of whether these technical security requirements can also be implemented using established best-practice approaches is investigated. The aim of this paper is to provide an overview of technical IT security requirements for banks in Austria, to discuss the respective topics addressed and to present possible measures for implementation.